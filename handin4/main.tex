\documentclass[a4paper]{article}

% --- Packages ---

\usepackage{a4wide}
\usepackage[utf8]{inputenc}
\usepackage{amsmath}
\usepackage{mathtools}
\usepackage{amssymb}
\usepackage[english]{babel}
\usepackage{mdframed}
\usepackage{systeme,}
\usepackage{lipsum}
\usepackage{relsize}
\usepackage{caption}
\usepackage{tikz}
\usepackage{tikz-3dplot}
\usetikzlibrary{shapes.geometric}
\usepackage{pgfplots}
\usepackage{pgfplotstable}
\pgfplotsset{compat=newest}%1.7}
\usepackage{harpoon}%
\usepackage{graphicx}
\usepackage{wrapfig}
\usepackage{subcaption}
\usepackage{authblk}
\usepackage{float}
\usepackage{listings}
\usepackage{xcolor}
\usepackage{chngcntr}
\usepackage{amsthm}
\usepackage{comment}
\usepackage{commath}
\usepackage{hyperref}%Might remove, adds link to each reference
\usepackage{url}
\usepackage{calligra}
\usepackage{pgf}

% --- Bibtex ---

%\usepackage[backend = biblar,]{bibtex}

%\addbibliografy(ref.bib)

% --- Commands --- 

\newcommand{\w}{\omega}
\newcommand{\trace}{\text{Tr}}
\newcommand{\grad}{\mathbf{\nabla}}
%\newcommand{\crr}{\mathfrak{r}}
\newcommand{\laplace}{\nabla^2}
\newcommand{\newparagraph}{\vspace{.5cm}\noindent}

% --- Math character commands ---

\newcommand{\curl}[1]{\mathbf{\nabla}\times \mathbf{#1}}
\newcommand{\dive}[1]{\mathbf{\nabla}\cdot \mathbf{#1}}
\newcommand{\res}[2]{\text{Res}(#1,#2)}
\newcommand{\fpartial}[2]{\frac{\partial #1}{\partial #2}}
\newcommand{\rot}[3]{\begin{vmatrix}\hat{x}&\hat{y}&\hat{z}\\\partial_x&\partial_y&\partial_z\\#1&#2&#3 \end{vmatrix}}
\newcommand{\average}[1]{\langle #1 \rangle}
\newcommand{\ket}[1]{|#1\rangle}
\newcommand{\bra}[1]{\langle #1|}


%  --- Special character commands ---

\DeclareMathAlphabet{\mathcalligra}{T1}{calligra}{m}{n}
\DeclareFontShape{T1}{calligra}{m}{n}{<->s*[2.2]callig15}{}
\newcommand{\crr}{\mathcalligra{r}\,}
\newcommand{\boldscriptr}{\pmb{\mathcalligra{r}}\,}


\title{Handin 4}
\author{Author : Andreas Evensen}
\date{Date: \today}

% --- Code ---

\definecolor{codegreen}{rgb}{0,0.6,0}
\definecolor{codegray}{rgb}{0.5,0.5,0.5}
\definecolor{codepurple}{rgb}{0.58,0,0.82}
\definecolor{backcolour}{rgb}{0.95,0.95,0.92}

\lstdefinestyle{mystyle}{
    backgroundcolor=\color{backcolour},   
    commentstyle=\color{codegreen},
    keywordstyle=\color{magenta},
    numberstyle=\tiny\color{codegray},
    stringstyle=\color{codepurple},
    basicstyle=\ttfamily\footnotesize,
    breakatwhitespace=false,         
    breaklines=true,                 
    captionpos=b,                    
    keepspaces=true,                 
    numbers=left,                    
    numbersep=5pt,                  
    showspaces=false,                
    showstringspaces=false,
    showtabs=false,                  
    tabsize=2
}

\lstset{style=mystyle}

\begin{document}

\maketitle
\noindent
Consider the dynamics of a strictly linear molecule $A-B-C$, in a 1-dimensional box.
It has a very stiff $A-B$ harmonic bond, and a very soft $B-C$ harmonic bond. The harmonic force constants are $k_{AB} = 4k_{BC}$ and $K_{AC} = 0$.
For completeness, we say that $r_{BC}^{eq} = 1.3 r_{AB}^{eq}$. Also, the center of mass of the molecule is confined in a very soft harmonic potential energy trap at $x = 0$, i.e. $k_{CM} << k_{BC}$.

\newparagraph
The probabilities of a configuration $i$ with energy $U_i$ can be assumed to fulfill $P(i)\propto e^{-\alpha U_i}$ where $\alpha > 0$ is a known constant.
Assume that the box is large enough such that one can ignore the boundaries, i.e. $x_{min} << x << x_{max}$.

\newparagraph
{\color{red}\textbf{1.} The only possible trail move is a displacement of an individual atom, e.g. A, B or C.}

\newparagraph
{\color{blue}\textbf{2.} There are three possible moves
\begin{enumerate}
    \item a displacement in an interval $[-h, h]$ of the atom $C$.
    \item a equal displacement in the interval $[-h, h]$ of the atoms $A$ and $B$.
    \item a small displacement in the interval $[-dh, dh]$ of the atoms $A$ and $B$ where they are moved in the opposite direction; $dh = \frac{1}{2}h$.
\end{enumerate}
}

\subsection*{1)}
Does the steps in {\color{red} 1.} and {\color{blue} 2.} span the same degrees of freedom for the atomic motion? Motivate your answer.

\newparagraph
\textbf{Answer: }They do span the same degree of freedom; in {\color{red} 1.} the only possible trail move is a displacement of an individual atom, while in {\color{blue} 2.} there are three possible moves, where two are coupled.

\newparagraph
If we restrict the motion to be either $h$ or $-h$ each particle has two degrees of freedom, and the total degrees of freedom is $2 + 2 + 2 = 6$.
If we again restrict the motion to be either $h$ or $-h$ for the atoms $A$ and $B$, and $dh$ or $-dh$ for the atoms $A$ and $B$; we can compute the number of possible moves.
For atom $A$ it can move either $h$ or $-h$, as well as $-dh$ and $dh$. However, this is coupled with the first second atom, atom $B$. And thus, the number of moves possible for the pair is $2 + 2 = 4$.
The last atom, atom $C$, can move either $h$ or $-h$, and thus has two degrees of freedom. Thus, in total for algorithm {\color{red} 1.} and {\color{blue} 2.} there are 6 degrees of freedom, and thus span the same degrees of freedom for the atomic motion.

\subsection*{2)}
Suggest an algorithm for {\color{red} 1.} that fulfills the detailed balance.

\newparagraph
\textbf{Answer: }By detailed balance one the probability for each individual move in one configuration is the same as the probability to move back to the same position in the new configuration:
\begin{align*}
    \mathcal{N}(\mathbf{x})\times \pi(\mathbf{x}\to\mathbf{x}') = \mathcal{N}(\mathbf{x}')\times \pi(\mathbf{x}'\to\mathbf{x}).
\end{align*}Here $\mathbf{x}$ represents a single particle in three dimension. A simple algorithm that would allow for this is the following:
\begin{enumerate}
    \item Pick one of the atom.
    \item Pick a distance in $x\in[-h, h]$.
    \item Make a trail move to the chosen atom to the new position. The acceptance is determined by $$a(\mathbf{x}\to\mathbf{x}')=\min\left(1,\frac{\mathcal{N}(\mathbf{x})}{\mathcal{N}(\mathbf{x}')}\right).$$
    \item If the move is accepted, move back to 1 and repeat, else repeat from 1 without moving the atom.
\end{enumerate}
Even though that $\mathcal{N}(\mathbf{x})$ and $\mathcal{N}(\mathbf{x}')$ is fairly heavy to compute, one has a working algorithm that preserves detailed balance.



\subsection*{3)}
In {\color{red} 1.}, if one decreases $h$ for the displacement trail step, the fraction of accepted displacement goes up. Why is this not necessarily more efficient?

\newparagraph
\textbf{Answer: }Even though the rate of which moves are accepted, the change in energy is smaller in each step.
Thus, more steps are usually required in order to change the system significantly. Say that you move atom $A$, which is bound by a stiff harmonic potential $U_{AB} = k_{AB}(r_{AB} - r_{AB}^{eq})$.
Each move of $A$ would almost always decrease the potential energy (in some cases it might increase slightly).
Thus, in order to reach the $0$ potential, one has to compute a high number of iterations in order to reach the equilibrium distance, whilst having larger step-sizes results in fewer computations for a more rapid change in the systems' configuration.


\subsection*{4)}
Discuss relative advantages and disadvantages of the two algorithms {\color{red} 1.} and {\color{blue} 2.}.

\newparagraph
\textbf{Answer: }As determined prior, the degrees of freedom in the two algorithms are the same.
The second algorithm implements what is discussed in the previous task with smaller step-size for the atoms $A$ and $B$, and they are moving together. 
But it also allows for translation of $A-B$. Which only affects the $B-C$ interaction which is weaker than the $A-B$ interaction.
Even though the degrees of freedom is higher, the number of iterations to reach a configuration which is lowest in potential energy, is higher for the {\color{blue} second algorithm}. However, the accuracy of this method is greater since the half-moves.
Thus, {\color{red} algorithm 1} will reach a configuration with lower energy than the initial state, but the {\color{blue} second algorithm} will reach a configuration with lower energy than the {\color{red} first algorithm} -- in more iterations.



\end{document}
 
