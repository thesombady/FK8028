\documentclass[a4paper]{article}

% --- Packages ---

\usepackage{a4wide}
\usepackage[utf8]{inputenc}
\usepackage{amsmath}
\usepackage{mathtools}
\usepackage{amssymb}
\usepackage[english]{babel}
\usepackage{mdframed}
\usepackage{systeme,}
\usepackage{lipsum}
\usepackage{relsize}
\usepackage{caption}
\usepackage{tikz}
\usepackage{tikz-3dplot}
\usetikzlibrary{shapes.geometric}
\usepackage{pgfplots}
\usepackage{pgfplotstable}
\pgfplotsset{compat=newest}%1.7}
\usepackage{harpoon}%
\usepackage{graphicx}
\usepackage{wrapfig}
\usepackage{subcaption}
\usepackage{authblk}
\usepackage{float}
\usepackage{listings}
\usepackage{xcolor}
\usepackage{chngcntr}
\usepackage{amsthm}
\usepackage{comment}
\usepackage{commath}
\usepackage{hyperref}%Might remove, adds link to each reference
\usepackage{url}
\usepackage{calligra}
\usepackage{pgf}

% --- Bibtex ---

%\usepackage[backend = biblar,]{bibtex}

%\addbibliografy(ref.bib)

% --- Commands --- 

\newcommand{\w}{\omega}
\newcommand{\trace}{\text{Tr}}
\newcommand{\grad}{\mathbf{\nabla}}
%\newcommand{\crr}{\mathfrak{r}}
\newcommand{\laplace}{\nabla^2}
\newcommand{\newparagraph}{\vspace{.5cm}\noindent}

% --- Math character commands ---

\newcommand{\curl}[1]{\mathbf{\nabla}\times \mathbf{#1}}
\newcommand{\dive}[1]{\mathbf{\nabla}\cdot \mathbf{#1}}
\newcommand{\res}[2]{\text{Res}(#1,#2)}
\newcommand{\fpartial}[2]{\frac{\partial #1}{\partial #2}}
\newcommand{\rot}[3]{\begin{vmatrix}\hat{x}&\hat{y}&\hat{z}\\\partial_x&\partial_y&\partial_z\\#1&#2&#3 \end{vmatrix}}
\newcommand{\average}[1]{\langle #1 \rangle}
\newcommand{\ket}[1]{|#1\rangle}
\newcommand{\bra}[1]{\langle #1|}


%  --- Special character commands ---

\DeclareMathAlphabet{\mathcalligra}{T1}{calligra}{m}{n}
\DeclareFontShape{T1}{calligra}{m}{n}{<->s*[2.2]callig15}{}
\newcommand{\crr}{\mathcalligra{r}\,}
\newcommand{\boldscriptr}{\pmb{\mathcalligra{r}}\,}


\title{Handin 3}
\author{Author : Andreas Evensen}
\date{Date: \today}

% --- Code ---

\definecolor{codegreen}{rgb}{0,0.6,0}
\definecolor{codegray}{rgb}{0.5,0.5,0.5}
\definecolor{codepurple}{rgb}{0.58,0,0.82}
\definecolor{backcolour}{rgb}{0.95,0.95,0.92}

\lstdefinestyle{mystyle}{
    backgroundcolor=\color{backcolour},   
    commentstyle=\color{codegreen},
    keywordstyle=\color{magenta},
    numberstyle=\tiny\color{codegray},
    stringstyle=\color{codepurple},
    basicstyle=\ttfamily\footnotesize,
    breakatwhitespace=false,         
    breaklines=true,                 
    captionpos=b,                    
    keepspaces=true,                 
    numbers=left,                    
    numbersep=5pt,                  
    showspaces=false,                
    showstringspaces=false,
    showtabs=false,                  
    tabsize=2
}

\lstset{style=mystyle}

\begin{document}

\maketitle
\section*{The system}
Suppose two systems:
{\color{blue}
\begin{align*}
    x_A(t + 1) = \frac{1}{2}x_A(t) - 3,
\end{align*}
}and {\color{red}
\begin{align*}
    x_B(t+1) = x_B(t)x_B(t) = x_B(t)^2.
\end{align*}
}

\subsection*{Question 1}
Sketch $x_i(t + 1)$ for the two systems in terms of $x_i$.

\newparagraph
\textbf{Answer: }We begin by rewriting $x_A$ in terms of $x_A(0)$,
\begin{align*}
    x_A(t + 3) &= \frac{1}{2}x_A(t + 2) - 3\\
    &= \frac{1}{2}\left(\frac{1}{2}x_A(t + 1) - 3\right) - 3\\
    &= \frac{1}{2}\left(\frac{1}{2}\left(\frac{1}{2}x_A(t)- 3\right) - 3\right)-3,
\end{align*}from this, we find that the recursion formula is given by:
\begin{align*}
    x_A(t) &= \left(\frac{1}{2}\right)^t x_A(0) - 3\left(2 - \left(\frac{1}{2}\right)^{t-1}\right).
\end{align*}
\begin{figure}[H]
    \centering
    \begin{tikzpicture}
        %\draw[variable = \x, domain = 0:10] plot (\x,\y) node[right] {$x_A(t)$};
        \draw (0,0) -- (10, 0) node[below] {$t$};
        \draw (0,0) -- (0, 3) node[left] {$x$};
        %\draw[scale=0.5, domain=0.01:10, smooth, variable=\x, blue] plot ({\x}, {pow(1/2, \x) - 6 + pow(3 / 2, \x - 1)});
    \end{tikzpicture}
\end{figure}The second system, $x_B$ can also be rewritten:
\begin{align*}
    x_B(t + 3) &= x_B(t + 2)^2\\
    &= x_B(t + 1)^4\\
    &= x_B(t)^6.
\end{align*}Thus, the recursion formula gives:
\begin{align*}
    x_B(t) = x_B(0)^{2t}
\end{align*}

\subsection*{Question 2}
How does the value of $\lim_{t\to\infty}x_i$ depend on the initial condition for $x(0)$ for the two systems.

\newparagraph
\textbf{Answer: }

\subsection*{Question 3}
For the first system, consider some initial condition $x_A(0)$ and a nearby condition, such that $\tilde{x}_a(0) = x_A(0) + \delta x_A(0)$.
Calculate the difference in the iterations.

\newparagraph
\textbf{Answer: }

\subsection*{Question 4}
The largest Lyapunov exponent is the exponential rate at which infinitesimally close initial conditions separate, i.e.
\begin{align*}
    \lambda = \lim_{t\to\infty}\frac{1}{t}\ln\abs{\frac{\delta x(t)}{\delta x(0)}}.
\end{align*}What is the Lyapunov exponent of system A? Note that $\delta x(0)$ is truly infinitesimal.

\newparagraph
\textbf{Answer: }

\end{document}
 
