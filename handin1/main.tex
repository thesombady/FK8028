\documentclass[a4paper]{article}
\usepackage{a4wide}
\usepackage[utf8]{inputenc}
\usepackage{amsmath}
\usepackage{mathtools}
\usepackage{amssymb}
\usepackage[english]{babel}
\usepackage{mdframed}
\usepackage{systeme,}
\usepackage{lipsum}
\usepackage{relsize}
\usepackage{caption}
\usepackage{tikz}
\usepackage{tikz-3dplot}
\usetikzlibrary{shapes.geometric}
\usepackage{pgfplots}
\usepackage{pgfplotstable}
\pgfplotsset{compat=newest}%1.7}
\usepackage{harpoon}%
\usepackage{graphicx}
\usepackage{wrapfig}
\usepackage{subcaption}
\usepackage{authblk}
\usepackage{float}
\usepackage{listings}
\usepackage{xcolor}
\usepackage{chngcntr}
\usepackage{amsthm}
\usepackage{comment}
\usepackage{commath}
\usepackage{hyperref}%Might remove, adds link to each reference
\usepackage{url}
\usepackage{calligra}

% Commands

\newcommand{\w}{\omega}
\newcommand{\curl}[1]{\mathbf{\nabla}\times \mathbf{#1}}
\newcommand{\grad}{\mathbf{\nabla}}
\newcommand{\dive}[1]{\mathbf{\nabla}\cdot \mathbf{#1}}
%\newcommand{\crr}{\mathfrak{r}}
\newcommand{\res}[2]{\text{Res}(#1,#2)}
\newcommand{\laplace}{\nabla^2}
\newcommand{\trace}{\text{Tr}}
\newcommand{\fpartial}[2]{\frac{\partial #1}{\partial #2}}
\newcommand{\rot}[3]{\begin{vmatrix}\hat{x}&\hat{y}&\hat{z}\\\partial_x&\partial_y&\partial_z\\#1&#2&#3 \end{vmatrix}}

% Special character commands
\DeclareMathAlphabet{\mathcalligra}{T1}{calligra}{m}{n}
\DeclareFontShape{T1}{calligra}{m}{n}{<->s*[2.2]callig15}{}
\newcommand{\crr}{\mathcalligra{r}\,}
\newcommand{\boldscriptr}{\pmb{\mathcalligra{r}}\,}
\newcommand{\nl}{\\\newline\noindent}



\title{FK8028: Handin1}
\author{Author : Andreas Evensen}
\date{Date: \today}
\definecolor{codegreen}{rgb}{0,0.6,0}
\definecolor{codegray}{rgb}{0.5,0.5,0.5}
\definecolor{codepurple}{rgb}{0.58,0,0.82}
\definecolor{backcolour}{rgb}{0.95,0.95,0.92}

\lstdefinestyle{mystyle}{
    backgroundcolor=\color{backcolour},   
    commentstyle=\color{codegreen},
    keywordstyle=\color{magenta},
    numberstyle=\tiny\color{codegray},
    stringstyle=\color{codepurple},
    basicstyle=\ttfamily\footnotesize,
    breakatwhitespace=false,         
    breaklines=true,                 
    captionpos=b,                    
    keepspaces=true,                 
    numbers=left,                    
    numbersep=5pt,                  
    showspaces=false,                
    showstringspaces=false,
    showtabs=false,                  
    tabsize=2
}

\lstset{style=mystyle}

\begin{document}

\maketitle

\section*{Integration}
Consider a simple spring (Hooke's law) which can be described by a set of differential equations:
\begin{align*}
    \dot{x} &= f(t, v) = v(t),\\
    \dot{v} &= g(t, x) = -\frac{k}{m}x(t)=-\w^2x(t).
\end{align*}Using Standard Euler one finds the following update scheme:
\begin{align}
    x_{n+1} &= x_n + \Delta t v_n,\label{eq: Stanrdard Euler}\\
    v_{n+1} &= v_n - \Delta t \w^2 x_n.\nonumber
\end{align}
\subsection*{a)}
One wants to show whether, or not, the Standard Euler scheme is time-reversible.
\nl
To show whether, or not, the scheme is time-reversible one needs to check the so called time-reversal condition, i.e. whether the scheme is invariant under the transformation $t + \Delta t\rightarrow t$.
We look at the coupled equations together:
\begin{align*}
    x_{n-1} &= x_n + (-\Delta t) v_{n-1}\\
    &= x_n - \Delta tv_{n-1},\\
    v_{n-1} &= v_n - (-\Delta t) \w^2 x_{n-1}\\
    &= v_n + \Delta t \w^2 x_{n-1}.
\end{align*}Since the system does not obey that $x_{n-1} = x_n - \Delta t v_n$ but instead $x_{n-1} = x_n - \Delta t v_{n-1}$, the scheme is not time-reversible and thus the time-reversal condition is not meet.


\subsection*{b)}
One wants to show whether, or not, Standard Euler is symplectic by calculating the Jacobian for a general Hamiltonian system and demonstrating whether that, in general, phase-space volume is conserved or not.
\nl
To check whether, or not, the system is symplectic we compute the Jacobian of the coupled equations \eqref{eq: Stanrdard Euler}:
\begin{align*}
    \begin{pmatrix}
        x_{n+1}\\
        v_{n+1}
    \end{pmatrix} = \underbrace{\begin{pmatrix}
        1 & \Delta t\\
        -\w^2\Delta t & 1
    \end{pmatrix}}_{\mathbf{A}}\begin{pmatrix}x_n\\v_n\end{pmatrix}.
\end{align*}The system is symplectic if the Jacobian of $\mathbf{A}$, $\mathbf{J}$ is symplectic, i.e. $\mathbf{J}\mathbf{w}\mathbf{J}^T=\mathbf{w}$, where $\mathbf{w} = ((0, 1)^T, (-1, 0)^T)^T$.
\begin{align*}
    \mathbf{J}\mathbf{w}\mathbf{J}^T &= \begin{pmatrix}
        0 & 1\\
        -\w^2 & 0
    \end{pmatrix}\begin{pmatrix}
        0 & 1\\
        -1 & 0
    \end{pmatrix}\begin{pmatrix}
        0 & -\w^2\\
        1 & 0
    \end{pmatrix}\\
    &=\begin{pmatrix}
        w^2 & 0 \\
        0 &-1
    \end{pmatrix}\neq\mathbf{w}.
\end{align*}Since the Jacobian is not symplectic, the system is not symplectic, i.e. the phase-space volume is not conserved.

\section*{Interactions}
Imagine a simulation of liquid water, H$_2$O$(l)$, with an intramolecular harmonic bond potentials, $(k^{\text{OH}}, r_{eq}^\text{OH})$ and $(k^{\text{HH}}, r_{eq}^{\text{HH}})$,
to define the intramolecular forces and an intramolecular Lennard-Jones potential $(\sigma^{OO}, \epsilon^{OO})$ defined the intermolecular forces.
The model contains $32$ molecules in gas phase. 

\subsection*{a)}
Write the full potential energy expression of the system with this approximate model, and be careful with the limits of all sums in the expression.
\nl
We want to write the full potential energy expression of the system with this approximate model, i.e. we want to write the potential energy of the system as a sum of the intramolecular and intermolecular potential potentials.
\begin{align*}
    U &= U_{\text{intra}} + U_{\text{inter}}\\
    U &= \underbrace{U_{\text{OH}} + U_{\text{HH}}}_{\text{intra}} + \underbrace{U_{\text{OO}}}_{\text{inter}}\\
    U &= \sum_{i = 1}^{32}\left\{k^{\text{OH}}\left(r_i^{\text{OH}} - r_{\text{eq}}^\text{OH}\right)^2 + k^{\text{HH}}\left(r_i - r_{\text{eq}}^{\text{HH}}\right)^2 + \sum_{j = 1, i \neq j}^{32}4\epsilon^{\text{OO}}\left[\left(\frac{\sigma^{\text{OO}}}{r_{i,j}}\right)^{12} - \left(\frac{\sigma^{\text{OO}}}{r_{i,j}}\right)^6\right]\right\}
\end{align*}where $r_{i,j}$ is the distance between to molecules $i$ and $j$, and $r_i$ is the distance between the two atoms in the $i$th molecule. The first potential term $k^{\text{OH}}\left(r_i - r_{\text{eq}}^{\text{OH}}\right)$ takes into account the two interactions that occur in each molecule. This is built into $k^{\text{OH}}$.

\subsection*{b)}
Discuss what physical interactions are missing, and how the model could be improved.

\vspace{0.5cm}\noindent
Assuming $k^{\text{HH}}$ is describing the covalent bonds that occur between two hydrogen atoms, and $k^{\text{OH}}$ is describing the covalent bonds that occur between an oxygen- and hydrogen-atom, one could improve the potential by additional terms such as:
\begin{enumerate}
    \item Bond angle potential: $U_{\text{bond angle}} = k_{\theta}^\text{OH}\left(\theta_{i,j} - \theta_{\text{eq}}\right)^2$.
    %\item Hydrogen-Hydrogen interaction in different molecules: (Van der Waals interaction)
    %\item Induced dipole-dipole interaction: $U_{\text{dipole}} = \frac{1}{4\pi\epsilon_0}\frac{1}{r_{i,j}^6}\left[\alpha_i\alpha_j - 3\left(\hat{r}_{i,j}\cdot\alpha_i\right)\left(\hat{r}_{i,j}\cdot\alpha_j\right)\right]$.
    \item Electric potential: $U_{\text{electric}} = \frac{1}{4\pi\epsilon_0}\frac{q_iq_j}{r_{i,j}}$.
    %\item Covalent bond potential: $U_{\text{covalent}} = k^{\text{HH}}\left(r_i^{\text{HH}} - r_{\text{eq}}^{\text{HH}}\right)^2 + k^{\text{OH}}\left(r_i^{\text{OH}} - r_{\text{eq}}^{\text{OH}}\right)^2$.
    \item Van der Waals interaction: $U_{\text{vdW}} = 4\epsilon^{\text{HH}}\left[\left(\frac{\sigma^{\text{HH}}}{r_{i,j}}\right)^{12} - \left(\frac{\sigma^{\text{HH}}}{r_{i,j}}\right)^6\right]$.
\end{enumerate}


\end{document}
 
